\documentclass[12pt]{article}
%\usepackage{tgschola}   % TeX Gyre Schola as the main font
\usepackage[utf8]{inputenc}
\usepackage[a4paper]{geometry}  \geometry{hmargin=1.3cm,vmargin=2.3 cm}
%\usepackage{fontspec}    % Allows you to load system fonts
%\setmainfont{TeX Gyre Schola} % Sets the main font for the document
\usepackage{xcolor}
\usepackage{amsmath}
\usepackage{amssymb}
\usepackage{amsfonts}
\usepackage{booktabs}
\usepackage{amsthm}
\usepackage{array}
\usepackage{epsfig}
\usepackage{caption}
\usepackage{empheq}
\usepackage{multicol}
\captionsetup{position=below}
\setlength{\parindent}{0pt}
\usepackage{graphicx}
\usepackage{color}
\usepackage{fancybox}
\usepackage{listings}
\usepackage{hyperref}
\usepackage{euscript}
\usepackage{float}
\usepackage{cite}

\setlength{\columnsep}{25pt} % Set the space between columns


\theoremstyle{plain}
\newtheorem{theorem}{Théorème}
\newtheorem{lemma}[theorem]{Lemme}
\newtheorem{corollary}[theorem]{Corollaire}
\newtheorem{proposition}[theorem]{Proposition}
\newtheorem{example}[theorem]{Exemple}

\theoremstyle{remark}
\newtheorem{remark}{Remarque}

\renewcommand*\contentsname{Table des matières}
\renewcommand{\listfigurename}{Liste des figures}
\renewcommand{\listtablename}{Liste des tableaux}

\newcommand{\vect}{\overrightarrow}

\begin{document}
		
	
	\hypersetup{pdfborder=0 0 0}
	\begin{titlepage}
		\newcommand{\HRule}{\rule{\linewidth}{0.5mm}}
		\begin{center}
			
			
			
			\vspace{1cm}
			
			\HRule \\ [0.8cm]
			{\Large {Rapport de stage :} }\\ [0.8cm]
			{\huge \bf Travail d'étude et de validation d'un code LES et comparaison avec la théorie de la turbulence de von Kármán.} \\ [0.4cm]
			\HRule \\ [2cm]
			{\large \textbf{Julien TENAUD}} \\ [0.3cm]
			{ Étudiants en $2^{\text{ème}}$ année à \textsc{l'enseirb-matmeca}} \\ [0.2cm]
			{ Spécialité : Mathématiques appliquées et mécanique} \\ [1cm]
			{\small sous la direction de }\\ [0.6cm]
			{\large \textbf{Pr. F.Bertagnolio}} \\ [0.3cm]
			{ Senior scientist at \textsc{dtu wind and energy systems}} \\ [1cm]
			{\small avec pour tuteur universitaire }\\ [0.6cm]
			{\large \textbf{M. N.Barral}} \\ [0.3cm]
			{ Maître de conférence à l'\textsc{enseirb matmeca}}\\ [0.3cm]
			
			\vfill
			
			\begin{minipage}[c]{.4\textwidth}
				\begin{center}
					\includegraphics[width=1\textwidth]{logo_emkk.jpg}
				\end{center}
			\end{minipage}
			
			\vfill
			
			
			{Année 2023/2024}
			
			
		\end{center}
	\end{titlepage}
	
	\newpage
	
	\section*{Sommaire}
	
	\noindent\rule{\textwidth}{0.3pt}
	\vspace{0.5cm}
	
	\shadowbox{%
		\begin{minipage}{1\textwidth}
			\tableofcontents
		\end{minipage}
	}

\clearpage
\section*{\Large Figures}
\noindent\rule{\textwidth}{0.3pt}
\vspace{0.5cm}

\listoffigures

\clearpage
\section*{\Large Tableaux}
\noindent\rule{\textwidth}{0.3pt}
\vspace{0.5cm}

\listoftables
	
\clearpage
	
	
	\section{Introduction}
	\noindent\rule{\linewidth}{2pt}
	\vspace{0.1cm}
	\subsection{Motivations de l'étude}
	
	Les éoliennes font maintenant partie de notre paysage. Elles sont de plus en présente sur terre (onshore) et maintenant en mer (offshore). En effet le besoin en énergie électrique est grandissant alors que nous somme en pleine crise écologique. Mais l'implentation et la recherche de performances des éoliennes n'est pas sans encombre. Exposition au vent, taille, raccordement au réseau... Les enjeux sont nombreux. Ce travaille s'incrit dans l'obectif de diminution du bruit généré par les pales d'éolienne. Le domaine d'étude et celui de l'aéro-acoustique. Dans de nombreux pays des normes gouvernementales sur le bruit émis \cite{impactacoustique2023} sont mises en place dans le bute de ne pas déranger les habitants aux alentours d'un parc éolien. Bien qu'aucune étude scientifique n'ai prouvé un quelconque effet nocif du bruit éolien sur la santé, de nombreuses plaintes ont été enregistrées de gens dérangés par ce bruit. Alors pour éviter que nos géants de métal ne devienne un ennemie publique, la réduction de leur effet sonore est nécessaire. \\
	
	Pour appréhender ce problème il faut d'abord comprendre comment les éoliennes génèrent du bruit. La grande majorité du son émis viens des pales de l'éolienne. En effet quand le flux d'air rencontre une pale il est perturbé et comme il y a interaction entre un fluide et une structure alors il y a apparition d'une couche limite. Dans notre cas cette dernière est turbulente. En effet le bout de pâle se déplace en moyenne à $80 m/s$ si on considère une éolienne classique. Dans l'étude de l'écoulement on à donc des Raynolds très élevés, de l'ordre de plusieurs milliers. La grande perturbation du flux d'air va engendrer la formation, tout autour de la pale, d'onde de pression qu'on appelle plus communément le bruit. Là où le son générer et le plus intense c'est au niveau du bord de fuite et plutôt en bout de pâle. Il y a un lien de causalité directe entre turbulence et son émis. En revanche il est considéré que le son émis n'a pas d'impacte sur la turbulence. C'est pourquoi la couche limite et le sillage turbulent des pâles sont étudiés.\\
	
	L'étude que j'ai mené s'incrit dans le travail de développement d'un code de calcul couplant l'aérodynamique et l'acoustique. Mon travail ne s'intéraissera qu'à la partie aérodynamique et plus précisément l'étude statistique d'une couche limite turbulente dans un canal parallélépipédique. Un code LES (Large Eddy Simulation) développé à \textit{DTU} permet la simulation de la turbulence notamment dans un canal à air. À noté que celui-ci est très utile vû le coût d'une expérience en soufflerie. Afin d'étudier et de valider les résultats obtenu après calcul nous effectuerons une analyse statistique notamment dans le domaine spectral de la turbulence et principalement des variables de vitesse. \\
	
	\subsection{Plan}
	Nous commencerons par détailler les outils mathématiques qui ont été utilisés pour étudier la turbulence. Nous présenterons ensuite brièvement la simulation effectuée et les modèles utilisés. Ils ne seront pas présentés en détaille car cela n'a pas fait l'objet du travail. L'idée était d'utiliser les résultats donnée afin de valider ces modèles et d'analyser la turbulence une fois validation. Nous présenterons par la suite les résultats obtenus selon plusieurs axes de recherche : 
	\begin{itemize}
		\item Validation des hypothèses de modèles semi-empiriques incluant la reconstitution du tenseur spectral avec les données a partir d'approche RANS.
		\item Étudier l'impact de certaines hypothèse faites dans les modèles semi-empiriques : 
		\begin{itemize}
			\item turbulence isotrope dans la couche limite
			\item distribution de l'énergie cinétique turbulente sur les 3 composantes et son évolution à travers la couche limite
			\item validité de l'hypothèse de turbulence figée (fluctuations turbulents constantes au cours du temps)
			\item évaluer des loi plus générales que la théorie de la turbulence isotropique.
		\end{itemize}
	\end{itemize}

\begin{center}
	\large {\bf{***}}
\end{center}

\vspace{0.3cm}
\section{Approche statistique et spectrale de la turbulence}
\noindent\rule{\linewidth}{2pt}
\vspace{0.1cm}

	Dans cette section nous introduiront mathématiquement certains outils d'étude statistique de la turbulence notamment dans le domaine spectrale \cite{mathieu2000introduction}. Nous nous pencherons sur les spectres en énergie et les fonctions de corrélation.
	
	\subsection{Fonction de corrélation et ses propriétés}
		
		Dans l'analyse d'un problème de turbulence on utilise la décomposition de Reynolds afin de décrire chaque grandeur avec un terme moyen et un terme de fluctuation, en prenant en exemple le champ de vitesse, 
		
		\begin{equation}
			U_i=<U_i> + u_i 	
		\end{equation}
	
		avec $<U_i>$ la vitesse moyenne (moyenne de Reynolds) et $u_i$ les fluctuations. \\
		Pour simplifier les équations nous utiliserons la notation d'Einstein de sommation implicite. \\
		La fonction de corrélation $R_{ij}$ se définie pour deux points $\textbf{x}$ et $\textbf{x'}$ ou deux temps $t$ et $t'$  tel que,
		
		\begin{equation}
		\begin{split}
			&R_{ij}(\textbf{x},\textbf{x'},t) = <u_i(\textbf{x},t) u_j(\textbf{x'},t)> \\
			&R_{ij}(\textbf{x},t,t') = <u_i(\textbf{x},t) u_j(\textbf{x},t')>
		\end{split}
		\end{equation}
	
		Dans le cas d'une turbulence homogène, c'est à dire qui conserve ses propriétés statistiques pour tout déplacement dans l'espace, alors $R_{ij}$ dépend uniquement de $\textbf{r}=\textbf{x}-\textbf{x'}$,
		
		\begin{equation}
			R_{ij}(\textbf{r},t) = <u_i(\textbf{x},t) u_j(\textbf{x'},t)>
			\label{eq:correlation}
		\end{equation}
	
		Nous pouvons définir la fonction d'autocorrelation,
		
		\begin{equation}
			R_{ii}(0,t) = <u_i(\textbf{x},t) u_i(\textbf{x},t)>
		\end{equation}
	
		\begin{proposition}[Indépendance statistique asymptotique]
			Quand $r\rightarrow\infty$, $R_{ij}(\textbf{r},t)\rightarrow0$. Plus les points considérés sont éloignés plus ils sont décorrélés.
		\end{proposition}
	
		\begin{proposition}
			Quand $t\rightarrow\infty$, $R_{ii}(0,t)\rightarrow0$. Un tourbillon se décorrèle de lui même au court du temps.
		\end{proposition}
	
		Définissons à présent une fonction qui nous intéressera tout au long de notre étude, le spectre qui est la transformée de Fourier tridimensionnelle de la fonction de corrélation,
		
		\begin{equation}
			\phi_{ij}(\textbf{k}, t) = \frac{1}{(2\pi)^3}\int R_{ij}(\textbf{r},t)e^{-ikr}d^3\textbf{r}
		\end{equation}
	 
	
		où $\textbf{k}$ est le vecteur d'onde. En prenant la transformée de Fourier inverse on peut obtenir :
		
		\begin{equation}
			R_{ij}(\textbf{r}, t) = \int \phi_{ij}(\textbf{k},t)e^{ikr}d^3\textbf{k}
			\label{spectra_space}
		\end{equation}
		
		Maintenant en imposant $i=j$ et $\textbf{r}=0$ dans (\refeq{spectra_space}) on obtient, 
		
		\begin{equation}
			<u_iu_i>=R_{ii}(0,t)=\int \phi_{ij}(\textbf{k},t)d^3\textbf{k}
			\label{verif}
		\end{equation}
	
		Alors il est possible de calculer le spectre en énergie nommé TKE par unité d'intensité de nombre d'onde (turbulent kinetic energy). 
		
		\begin{equation}
			E(\kappa)=2\pi \kappa^2\phi_{ii}~~[m^{3}.s^{-2}]
			\label{energie_spectra}
		\end{equation}
	
		où $\kappa = |\textbf{k}|$. On peut tracer le spectre d'énergie turbulente $log(E)=f(log(k))$. Ce spectre vérifie d'après (\refeq{energie_spectra}) et (\refeq{verif}),
		
		\begin{equation}
			<u_iu_i>=2\int_{0}^{\infty}E(\kappa,t)d\kappa
		\end{equation}
	
	\subsection{Spectre unidimensionnel et propriété de turbulence figée}
	
		On considère l'espace représenté par le repère cartésien $(O,x_1,x_2,x_3)$. L'écoulement turbulent étudié est homogène dans la direction \textbf{x} de l'espace. Alors on défini la corrélation spatial à deux points \\$R_{ij}(r_1, x_2,x_3,x_2',x_3',t) = <u_i(\textbf{x},t)u_j(\textbf{x'},t)>$ où $r_1=x_1-x_1'$. Le spectre spatial unidimensionnel est donc donné par,
		
		\begin{equation}
			\phi_{ij}^{[1]}(k_1,x_2,x_3,t)=\frac{1}{2\pi}\int_{-\infty}^{\infty}R_{ij}(r_1,x_2,x_3,x_2,x_3,t)e^{-ik_1r_1}dr_1
		\end{equation}
	
		où on a considéré $\textbf{x}$ et $\textbf{x'}$ sur une même ligne de direction $x_1$. Alors, si on considère $x_1'=x_1$ ($r_1=0$) et que $i=j$ on peut montrer avec une transformée de Fourier inverse que:
		
		\begin{equation}
			<u_iu_i>=2\int_{0}^{\infty}\phi_{ii}^{[1]}(k_1,x_2,x_3,t)dk_1
		\end{equation}
	
		À présent considérons la fonction de corrélation en temps $R_{ij}(\textbf{x},\tau)=<u_i(\textbf{x},t)u_i(\textbf{x},t')>$ au même point mais à deux temps différent. Il est alors possible d'obtenir la fonction spectral en fréquence :
		
		\begin{equation}
			\psi_{ij}(\omega,\textbf{x})=\frac{1}{2\pi}\int_{-\infty}^{\infty}R_{ij}(\textbf{x},\tau)e^{i\omega\tau}d\tau
		\end{equation}
	
		où $\omega$ représente les fréquences. On peut obtenir de même : 
		
		\begin{equation}
			<u_iu_i>=\int_{-\infty}^{\infty}\psi_{ij}(\omega,\textbf{x})d\omega
		\end{equation}
		
		
		Expérimentalement, les mesures de vitesse d'écoulement à plusieurs points au même moment ne sont pas facile à effectuer. On aimerai bien pouvoir relever les caractéristique de la turbulence en un point à plusieurs temps et en déduire des informations sur l'évolution des structures spatial. C'est pourquoi on fait l'hypothèse suivante:
		
		\begin{proposition}[Hypothèse de turbulence figée ou de Taylor]
			On considère que les structures spatiales d'intéret ne change pas de manière signifiante durant le temps de leur passage le long de la ligne de mesure. 
			\label{prop:turb-fig}
		\end{proposition}
	
		Ainsi, on suppose que l'échelle de temps liée à la dynamique des tourbillons est grande devant le temps de parcours du fluide dans la cavité de mesure. Dans le cas d'un écoulement de direction préférentiel $x_1$ cela peut s'exprimer tel que :
		
		\begin{equation}
			u_i=u_i(x_1-<U_1>t,x_2,x_3)
		\end{equation}
	
		Un des aspect de cette étude sera de vérifié si cette hypothèse est vérifiée dans le cas des calculs LES effectués. \\
		
		Dans les écoulements stationnaire où cette hypothèse est vérifié on peut alors montrer que :
		
		\begin{equation}
			\phi_{ij}^{[1]}(k_1,x_2,x_3)=|<U_1>|\psi_{ij}(<U_1>k_1,x_2,x_3)
			\label{fig:turb-fig}
		\end{equation}
	
		Cette expression relie le spectre en fréquence et le spectre unidimensionnel en nombre d'onde. Cela met en évidence la relation $\omega=<U_1>k_1$ entre la fréquence et le nombre d'onde unidimensionnel pour des composantes de Fourier convéctés à une vitesse moyenne $<U_1>$. 
		
		
\begin{center}
	\large {\bf{***}}
\end{center}

\vspace{0.3cm}
\section{Résultats théorique à partir du modèle de von Kármán de la turbulence}
\noindent\rule{\linewidth}{2pt}
\vspace{0.1cm}
		
	Dans cette section nous allons définir certaines des expressions des spectres en utilisant les résultat de von Kármán en 1948 \cite{vonkarman1948} sans pour autant en dresser une liste exostive. Par la suite l'objectif sera de comparer le modèle de von Kármán théoriques avec les données issues des calculs LES à Reynolds variables. \\
	
	Le modèle utiliser par von Kármán vient de l'expression du spectre d'énergie (\refeq{energie_spectra}) sous l'hypothèse que la turbulence est isotropique \cite{wilson1998turbulence},
	
	\begin{equation}
		E(\kappa) = \frac{4\Gamma(\nu + 5/2)}{3\sqrt{\pi}\Gamma(\nu)}\frac{\sigma^2\kappa^4l^5}{(1+\kappa^2l^2)^{\nu + 5/2}}
		\label{energie_spectra}
	\end{equation}

	où $\sigma^2$ est la variance, $\kappa$ est l'intensité du vecteur d'onde, $l$ "the length scale", $\Gamma()$ est la fonction gamma et $\nu$ est un paramètre qui définie la loi de puissance dans la zone inertiel ($\kappa l \gg 1$). Pour notre étude il sera fixé à $1/3$ afin de retrouver la loi de puissance de Kolmogorov de $-5/3$ \cite{kolmogorov1991local}. \\

	De cette expression il est possible d'obtenir les spectres 'streamwise' avec une intégration par rapport à $k_2$ et $k_3$ des spectres d'énergie $E_{11}(\kappa)$, $E_{22}(\kappa)$ et $E_{33}(\kappa)$:
	
	\begin{equation}
		\phi_{11}^1(k_c,z) = \frac{36\Gamma(17/6)\sigma_1l}{55\sqrt{\pi}\Gamma(1/3)}\frac{1}{(1 + (k_cl)^2)^{5/6}}
		\label{eq:phi_vk_1}
	\end{equation}

	\begin{equation}
		\phi_{22}^1(k_c,z) = \frac{6\Gamma(17/6)\sigma_2l}{55\sqrt{\pi}\Gamma(1/3)}\frac{(3+8(k_cl)^2)}{(1 + (k_1l)^2)^{11/6}}
		\label{eq:phi_vk_2}
	\end{equation}

	\begin{equation}
		\phi_{33}^1(k_c,z) = \frac{6\Gamma(17/6)\sigma_3l}{55\sqrt{\pi}\Gamma(1/3)}\frac{(3+8(k_cl)^2)}{(1 + (k_1l)^2)^{11/6}}
		\label{eq:phi_vk_3}
	\end{equation}

	avec $\sigma_i = <u_iu_i>$, $l=C\frac{k_T^{3/2}}{\varepsilon}$, $k_c = \frac{\omega}{U_c}$.\\
	
	Ce sont ces trois spectres que nous chercherons à comparer avec ceux obtenu à partir des données issus des modèles LES dans la section $9$. Nous expliquerons donc dans la dites section comment nous obtenons les valeurs de $l$ et $k_c$.
	
	
		 
\begin{center}
	\large \bf{***}
\end{center}

\vspace{0.3cm}	
\section{Modèles WMLES et WRLES}
\noindent\rule{\linewidth}{2pt}
\vspace{0.1cm}

Pour commencer à décrire l'étude qui ont été menée et les résultats qui ont été obtenues durant ce stage nous devons tout d'abord parler des modèles LES (large eddy simulation) de simulation numérique qui nous on permis l'aquisition de données sur la turbulence. Nous ne détaillerons pas mathématiquement et numériquement ces modèle de simulation d'écoulement turbulent car ce n'est pas l'objet de ce rapport. \\

Les données analysés par la suite viennent de la simulation d'un écoulement turbulent en régime établie dans un canal droit parallélépipédique ("channel flow") (Figure \ref{fig:channel_flow}). Les bord nord et sud par rapport au sens de l'écoulement sont des mures (condition 'Wall') et le reste des bords sont périodiques. Par la suite nous dénommerons l'axe de l'écoulement "streamwise" ($\vect{x}$), l'axe normal au mure "wall-normal" ($\vect{z}$) et l'axe normal au deux précédent "spanwise" ($\vect{y}$). \\

\begin{figure}[h!]
	\begin{center}
		\includegraphics[width=0.62\linewidth]{../../report/referance/channel_flow.png}
		\caption{Schéma de la simulation "channel flow"}
		\label{fig:channel_flow}
	\end{center}
\end{figure}

Sur la Figure \ref{fig:channel_flow} une seul ligne de mesure dans les direction $x$ et $y$ est représenter par soucis de clarté. Dans nos simulation nous en avons utiliser entre 5 et 10 selon le modèle. Le détaille des maillages et des paramètres utilisés dans les simulations est présent dans la section suivante. \\

La Large Eddy Simulation (LES) est une méthode de simulation numérique utilisée pour résoudre les équations de Navier-Stokes en modélisant les écoulements turbulents. Contrairement aux méthodes plus simplifiées comme les Reynolds-Averaged Navier-Stokes (RANS), qui modélisent toute la turbulence, la LES se concentre sur une approche plus fine. Elle consiste à résoudre explicitement les grandes structures tourbillonnaires de l’écoulement (grandes échelles) tout en modélisant les plus petites structures (petites échelles) qui ne peuvent être capturées directement. \\
La méthode LES repose sur un filtrage spatial des équations de Navier-Stokes. Ce filtrage sépare les grandes échelles (résolues) des petites échelles (modélisées via un modèle de sous-mailles). L’idée principale est que les grandes échelles de turbulence sont spécifiques à l’écoulement et doivent être directement simulées, tandis que les petites échelles, supposées plus universelles, peuvent être modélisées avec un modèle de turbulence de sous-mailles. Au niveau des parois la résolution des petites échelles devient nécessaire et donc les méthode LES sont assez coûteuses même si le coup varie beaucoup en fonction du sous modèle choisi. \\

Dans notre étude nous avons utilisé deux sous modèles de type LES : le Wall-Resolved LES (WRLES) et le Wall-Modeled LES (WMLES). \\
Le modèle WRLES résout explicitement toutes les structures turbulentes, y compris celles présentes dans la couche limite proche des parois. Cela signifie qu’une résolution extrêmement fine est nécessaire à proximité de la paroi pour capturer les petites échelles de la turbulence. En pratique, cette approche est coûteuse, car la taille des mailles doit être très petite dans cette région, d’autant plus que le nombre de Reynolds est élevé. \\
Le modèle WMLES est une approche hybride dans laquelle on utilise un modèle de paroi (wall model) pour modéliser l’écoulement turbulent à proximité des parois, tandis que les grandes échelles turbulentes loin des parois sont résolues de manière explicite par la LES. En d'autres termes, la WMLES évite la nécessité de résoudre explicitement les petites structures turbulentes dans la couche limite, ce qui permet d'utiliser une grille plus grossière près des parois, réduisant ainsi les coûts de calcul.

\begin{center}
	\large \bf{***}
\end{center}

\vspace{0.3cm}
\section{Analyse simple des variables de vitesse turbulente afin de valider les jeux de données}
\noindent\rule{\linewidth}{2pt}
\vspace{0.1cm}

Commençons par analyser simplement les données de sortie obtenu afin de voir si les résultats correspondent au attente d'une simulation d'écoulement turbulent dans un "channel flow". Les paramètres utilisées pour les trois simulations dans le tableau \ref{tab:parameters}: \\

\begin{table}[!h]
	\begin{tabular}{l | c | c | c}

		Grandeurs & WRLES $Re_{\tau}=395$ & WMLES $Re_{\tau}=1000$ & WRLES $Re_{\tau}=1000$\\ \hline \hline
		Dimension du canal ($x\times y \times z$) &  $2\pi\times\pi\times2$ & $4\pi\times1.5\pi\times2$ & $4\pi\times1.5\pi\times2$\\
		Masse volumique fluide &  $\rho = 1.0$ & $\rho = 1.0$ & $\rho = 1.0$ \\
		Vitesse à "l'infinie" & $u_{\infty}=1.0$ & $u_{\infty}=1.0$ & $u_{\infty}=1.0$  \\
		Grille ($N_x\times N_y \times N_z$) & $256\times128\times1936$ & $192\times96\times96$ & $640\times257\times320$ \\
		Nombre d'iteration & $N_{iter}=50000$ & $N_{iter}=75000$ & $N_{iter}=150000$ \\
		Pas de temps & $\delta t=0.01$ & $\delta t=0.01$ & $\delta t=0.004$ \\
		Nombre de Reynolds & $Re = 13800.0$ & $Re = 40000.0$ & $Re = 40000.0$ \\
		Reynolds de frottement & $Re_{tau} = 388.97$ & $Re_{tau} = 968.55$ & $Re_{tau} = 980.86$ \\
		Viscosité dynamique& $\mu = 0.000145$ & $\mu = 5e-05$ & $\mu = = 5e-05$ \\
		Viscosité cinématique & $\nu = 0.000145$ & $\nu = 5e-05$ & $\nu = 5e-05$ \\
		Vitesse de frottement & $u_{\tau} = 0.0564$ & $u_{\tau} = 0.0484275$ & $u_{\tau} = 0.0484275$ \\
		\hline
	\end{tabular}
	\caption{Paramètres des simulations numériques LES effectuées}
	\label{tab:parameters}
\end{table}


Afin de valider les résultats des calculs LES nous avions à disposition des données issues de calculs DNS (Direct Navier-Stokes) effectué auparavant \cite{lee2015direct}. Aussi pour le cas $Re = 395$ un calcul avec un modèle RANS à été effectué. Pour que les résultats soit comparable les données on été relevées aux même auteurs adimensionnelles que les calcules DNS (tab. \ref{tab:zplus}). \\

\begin{table}[!h]
\begin{tabular}{l | c }

	Modèles & localisation normal ($z^+$) \\	\hline	\hline
	WRLES $Re_{\tau}=395$  & $5,~20,~40,~98,~151,~199,~251,~302,~392,~$ \\
	WMLES $Re_{\tau}=1000$ & $151,~199,~251,~302,~380.04,~392,~503.63,~638.27,~762.79,~990.67$ \\
	WRLES $Re_{\tau}=1000$ & $~20,~40,~98,~151,~199,~251,~302,~380.04,~392,~503.63,~638.27,~762.79,~990.67$ \\
	\hline
\end{tabular}
	\caption{Hauteurs adimensionnelles auxquelles ont été relevées les quantités de vitesse}
	\label{tab:zplus}
\end{table}

Les grandeurs relevées auxquelles nous nous sommes intéressé dans ce travail sont les vitesses "streamwise" ($u_x, u_1 \text{ ou } u$), "spanwise" ($u_y, u_2 \text{ ou } v$) et wall-normal ($u_z, u_3 \text{ ou } w$). Nous noterons par la suite les vitesses moyennes turbulentes avec des majuscules ($U,~V,~W$) et les fluctuations turbulentes de vitesse avec des minuscules ($u,~v,~w$).\\

Regardons pour commencer les profiles de vitesse moyenne et des composantes du tenseur de Reynolds ainsi que le profile d'énergie cinétique turbulente, $k_T=\frac{1}{2}(u^2+v^2+w^2)$ tracé en {\bf Figures \ref{fig:mean-vel}} et \ref{fig:fluct-vel}. Toutes ces vitesses ont été addimentionnées par la vitesse de frottement $u_{\tau}$. Les vitesses moyennes sont obtenues en faisant une moyenne à la fois en temps et en espace. Pour obtenir les fluctuations turbulentes on soustrait cette moyenne aux données brutes.\\
Les profiles de vitesse "steamwise" sont comparables aux données DNS et RANS. Par ailleurs les vitesses "spanwise" et "wall-normal" sont petites devant cette dernière et le comparatif est donc un peut moins idéal. On observe un certain décalage dans la comparaison des composantes du tenseur de Reynolds pour celles incluant $u_2$ et $u_3$. Malgré ces différence dû au modèle utilisé les profiles restent semblables et les données sont donc utilisable pour une analyse plus approfondie. La même étude à été mener pour les deux modèles à Reynolds 1000.


\begin{figure}[h!]
	\begin{center}
		\includegraphics[width=0.8\linewidth]{../../output/figures/channel_wrles_retau395/split_time/RANS/RANS_LES_MOSER_profiles_all.png}
		\caption{Profiles de vitesse moyenne addimensionnée et d'énergie cinétique turbulente. Les données ont été comparées aux profiles des données DNS et RANS quand cela était possible.}
		\label{fig:mean-vel}
	\end{center}
\end{figure}

\begin{figure}[h!]
	\begin{center}
		\includegraphics[width=0.8\linewidth]{../../output/figures/channel_wrles_retau395/split_time/RANS/var_velocity_profiles_all.png}
		\caption{Profiles de fluctuation. Les données ont été comparées au profiles des données DNS.}
		\label{fig:fluct-vel}
	\end{center}
\end{figure}

\pagebreak

Maintenant que les jeux de données semble correcte nous allons pouvoir s'interesser à l'hypothèse de turbulence figée (prop. \ref{prop:turb-fig}). Dans la théorie de résolution cette hypothèse est fait on veut donc voir si les résultats du calcul la confirme. 


\begin{center}
	\large {\bf{***}}
\end{center}

\vspace{0.3cm}
\section{Validation de l'hypothèse de turbulence figée}
\noindent\rule{\linewidth}{2pt}
\vspace{0.1cm}

Pour cette section nous nous appuierons sur l'équation \ref{fig:turb-fig} qui traduit mathématiquement l'hypothèse de turbulence figée. Notons que les figures données par la suite correspondrons au modèle WRLES-$Re_{\tau}395$  Nous voulons vérifier que le spectre calculer par rapport à l'espace est à une constante près égale au spectre calculer par rapport au temps. Cette constante correspond à la vitesse de convection moyenne du fluide dans le sens de son écoulement dans le canal qu'on notera $U_c$. Il faut donc tout d'abord trouver cette constante. Pour cela nous allons calculer la fonction d'autocorrelation bidimensionnelle $R(\delta t, \delta x)$ et tracer ses iso-valeur sur le plan ($\delta x, \delta t$) ({\bf Figure \ref{fig:corr-2d}}). La pente définie par la direction des ellipses correspond à $\frac{1}{U_c}$.

\begin{figure}[h!]
	\begin{center}
		\includegraphics[width=0.9\linewidth]{../../output/figures/channel_wrles_retau395/split_time/frozen_turbulence/correlation2D/u1.png}
		\caption{Iso-valeur de la fonction d'autocorrelation 2D selon la direction $x$ et le temps $t$ à différentes hauteurs $z^+$. Obtention de la valeur de $U_c$ par approche linéaire de la direction des ellipses.}
		\label{fig:corr-2d}
	\end{center}
\end{figure}

Les valeurs de $U_c$ trouver doivent se rapprocher des valeurs moyennes de vitesse "streamwise" $U_1$ (\textcolor{blue}{cf. fig ... annexe}).\\

A présent on calcule le spectre selon $k$ (espace) et le spectre
selon $\omega$ (temps) en utilisant la méthode de Welch \cite{welch1967spectra} avec une fenêtre de "hann". En regardant la {\bf Figure \ref{fig:spectra-space-time}} on peu conclure que l'équation \ref{fig:turb-fig} est vérifié et que l'hypothèse de turbulence figée est donc bien admissible dans notre simulation.

\begin{figure}[h!]
	\begin{center}
		\includegraphics[width=0.85\linewidth]{../../output/figures/channel_wrles_retau395/split_time/frozen_turbulence/power_spectra/u1.png}
		\caption{Comparaison des spectres de puissance en temps ($F$) et dans la direction $x$ ($P$) tracer en fonction de $k_x$ le nombre d'onde "streamwise". En pointillé (- - -) la pente $k_x^{-5/3}$ vérifiant la décroissance de Kolmogorov \cite{kolmogorov1991local}}
		\label{fig:spectra-space-time}
	\end{center}
\end{figure}
\pagebreak

\begin{center}
	\large \bf{***}
\end{center}

\vspace{0.3cm}
\section{Détermination du coefficient de décorrélation}
\noindent\rule{\linewidth}{2pt}
\vspace{0.1cm}

Une grandeur statistique intéressante à regarder est la corrélation (Équation \ref{eq:correlation}) afin de voir son évolution en fonction de $r=\delta x$ par rapport à un point de référence.\\
Pour points de référence nous prendrons les premiers point à $x=\delta x$ à quatre hauteurs différentes. En {\bf Figure \ref{fig:space_spectra}} vous trouverez les courbes d'auto-corrélation ($R_{UU}$, $R_{VV}$, $R_{WW}$) de vitesses dans les trois directions selon les points de relever "streamwise".  \\


\begin{figure}[h!]
	\begin{center}
		\includegraphics[width=0.9\linewidth]{../../output/figures/channel_wrles_retau395/split_time/space_correlation/streamwise.png}
		\caption{Corrélation streamwise des grandeur de vitesse dans les trois directions}
		\label{fig:space_spectra}
	\end{center}
\end{figure}

À présent nous cherchons à déterminer le coefficient de décorrélation qu'on nommera $\gamma$. Si on suppose que l'écoulement est isotrope, homogène dans la direction de détermination du coefficient et que l'hypothèse de turbulence figée est vérifiée alors on peut déterminer l'équation suivante: 

\begin{equation}
	\phi(r, \omega) = \frac{e^{-|\gamma k_c r| + jk_cr}}{U_c}\phi(\omega)
	\label{eq:funct}
\end{equation} 

avec $r = \delta x$, $k_c=\omega/U_c$ le nombre d'onde "streamwise", $\gamma$ le coefficient de décorrélation et $j$ le nombre imaginaire tel que $j^2=-1$. Le spectre $\phi(\omega)$ à été calculé avec la méthode de Welch et le spectre $\phi(\delta x, \omega)$ à été calculé en appliquant une transformé de Fourier à la fonction de corrélation bidimensionnelle $R_{ii}(r,\tau)$ en fonction du temps.\\
Ainsi il est possible de tracer $\beta = |\gamma k_c r|$ en fonction de $k_c$ (ou $\omega$) ({\bf Figure \ref{fig:gamma_w}}) ou $r$ et de déterminer $\gamma$. Dans la littérature on trouve des coefficients de décorrélation de l'ordre de $0.3$.



\begin{figure}[h!]
	\begin{center}
		\includegraphics[width=0.9\linewidth]{../../output/figures/channel_wrles_retau395/split_time/gamma/gamma_u1_w.png}
		\caption{Tracé de la fonction $\beta$ en fonction de $k_c$ pour la vitesse $u_1$ à quatre hauteurs dans le canal et pour quatre différents $r$. Les pointillés représente la pente approchée de la courbe}
		\label{fig:gamma_w}
	\end{center}
\end{figure}

On a alors le coefficient de pente calculer en {\bf Figure \ref{fig:gamma_w}}, $c=\gamma r / U_c$. On peut donc déterminer le coefficient de décorrélation dont on a donné les valeurs en {\bf Figures \ref{fig:gamma_w_view}} pour les trois composantes de vitesse. \\

Il semblerait que nous n'obtenions pas le coefficient attendu, notamment pour des points très rapprochés ($r$ petits). Cela signifie sûrement que le modèle pris pour déterminer l'equation \ref{eq:funct} ne s'applique pas dans notre cas. Par exemple sur les calculs LES l'isotropie n'est pas vérifié.

\begin{figure}[H]
	\begin{center}
		\includegraphics[width=0.8\linewidth]{../../output/figures/channel_wrles_retau395/split_time/gamma/gamma_view_w_all_mod.png}
		\caption{Tracé de la fonction $\beta$ en fonction de $k_c$ pour la vitesse $u_1$ à quatre hauteurs dans le canal et pour quatre différents $r$. Les pointillés représente la pente approchée de la courbe}
		\label{fig:gamma_w_view}
	\end{center}
\end{figure}


\begin{center}
	\large \bf{***}
\end{center}




\vspace{0.3cm}
\section{Comparaison des spectres avec la théorie de von Kármán}
\noindent\rule{\linewidth}{2pt}
\vspace{0.1cm}

Nous voulons maintenant comparer les résultats spectrales obtenu avec les modèles LES et les spectres obtenus avec la théorie de von Kármán explicités en section 3. Nous comparons aussi ces résultats au données DNS \cite{lee2015direct}. \\
La théorie de von Kármán s'appui sur l'hypothèse d'isotropie qui n'est pas valide dans le cas des modèles LES. Nous voulons donc quantifié les écartes en énergie que la théorie de von Kármán à avec les modèles LES et DNS. Pour cela nous utilisons les équations \ref{eq:phi_vk_1}, \ref{eq:phi_vk_2} et \ref{eq:phi_vk_3} pour calculer les spectres de densité d'énergie selon la théorie de von Kármán. Les spectres issus des données LES sont quant à eux calculés grâce à la méthode de Welch en espace ({\bf Figure \ref{fig:vk_spectra}}). La pente de coefficient $-\frac{5}{3}$ à été tracé en pointillé sur la {\bf Figure \ref{fig:vk_spectra}} afin de vérifier la théorie de Kolmogorov. \\

Les résultats LES et DNS sont quasiment similaire. Cela confirme que notre calcul LES est bon. Pour autant un écart assez conséquents ce remarque avec la théorie de von Kármán. Comme dans la théorie de von Kármán le spectre s'étand sur un nombre d'onde infinie nous remarquons que les spectres théorique sont bien plus étendus. Nous voulons donc s'avoir si la quantité d'énergie, c'est à dire mathématiquement "integral length scale", sont équivalentes. En {\bf Figure \ref{fig:vk_integral_scale}} on constate que le rapport de ces grandeurs sont entre 2 et 3. La théorie de von KKármán n'est donc pas une parfaite approximation des phénomènes de turbulence en canal pour autant dans certain cas ce modèle théorique peut rester une bonne approximation. 
	

\begin{figure}[H]
	\begin{center}
		\includegraphics[width=0.9\linewidth]{../../output/figures/channel_wrles_retau395/split_time/von_karman/von_karman_spectra_.png}
		\caption{Comparaison des spectre de densité d'énergie issus de la théorie de von Kármán avec ceux issu des données LES et DNS.}
		\label{fig:vk_spectra}
	\end{center}
\end{figure}


\begin{figure}[H]
	\begin{center}
		\includegraphics[width=0.7\linewidth]{../../output/figures/channel_wrles_retau395/split_time/von_karman/integral_lenght_scale.png}
		\caption{Rapport des échelles de longueurs intégrales entre les données LES et la théorie von Kármán}
		\label{fig:vk_integral_scale}
	\end{center}
\end{figure}
	


\begin{center}
	\large \bf{***}
\end{center}

\vspace{0.3cm}
\section{Conclusion}
\noindent\rule{\linewidth}{2pt}
\vspace{0.1cm}

Notre travaille à permis dans un premier temps de valider les résultats du code LES. Par la suite nous avons vérifié une hypothèse forte, celle de turbulence figée. Avec ces deux résultats cela permet d'être confiant quant à l'utilisation des modèles WRLES et WMLES dans le cas d'une simulation en soufflerie comme celle détaillée dans le rapport. On à pu par la suit chercher à retrouver le coefficient de décorrélation. Cela n'a pas aboutit à de résultats convainquant mais pour autant il reste inintéressant vis à vis de la validité de l'équation \ref{eq:funct} dans notre cas. Pour finir nous avons pu observer les écarts entre la théorie de von Kármán et les résultats numériques LES et DES dans notre configuration. À notre connaissance, cela n'avait pas été fait auparavant et on a pu constate que dans notre cas la théorie de von Kármán ne permet pas de recouvrir toutes l'énergie que nous obtenons avec les résultats de simulation. \\ 
Dans ce rapport seulement les résultats du modèle WRLES - Retau395 on été montrés car ils ont représenté la majorité du travail effectué. Pour autant les calculs et des résultats on aussi été obtenus sur les modèles à $Re_{\tau}=1000$. De plus une comparaison entre les deux modèles WRLES et WMLES pourrait faire l'objet de suite à ce travail. En effet au niveau temps de calcul le modèle WMLES est bien plus rapide. Il faudrait donc voir si les résultats donnés par celui-ci sont assez satisfaisant par rapport au modèle WMLES afin de détermine lequel est le plus intéressant de choisir dans le cas de notre simulation. \\
L'objectif à terme est de pouvoir utiliser le code établie durant ces trois mois pour effectuer des calcules similaires sur une simulation de pâle d'éolienne en soufflerie. En revanche dans ce nouveaux cas l'écoulement n'est plus homogène dans aucune direction ce qui rend les calculs plus compliqués. Le code à été écrit de sort à ce qu'il soit modulaire et donc facilement malléable. Il serait donc assez simple, au niveau de la programmation, d'adapter le code au cas d'une pâle. En revanche il faudrait faire attention aux calculs qui n'ont plus de sens dans ce cas précis.

\pagebreak

\bibliographystyle{plain}
\bibliography{biblio.bib}	
	
\end{document}