\documentclass[twocolumn]{article}
\usepackage[utf8]{inputenc}
\usepackage[a4paper]{geometry}  \geometry{hmargin=1.0cm,vmargin=2 cm}
\usepackage{xcolor}
\usepackage{amsmath}
\usepackage{amssymb}
\usepackage{amsfonts}
\usepackage{booktabs}
\usepackage{amsthm}
\usepackage{array}
\usepackage{epsfig}
\usepackage{caption}
\usepackage{empheq}
\usepackage{multicol}
\captionsetup{position=below}
\setlength{\parindent}{0pt}
\usepackage{graphicx}
\usepackage{color}
\usepackage{fancybox}
\usepackage{listings}
\usepackage{hyperref}
\usepackage{euscript}
\usepackage{float}
\usepackage{cite}

\setlength{\columnsep}{25pt} % Set the space between columns


\theoremstyle{plain}
\newtheorem{theorem}{Théorème}
\newtheorem{lemma}[theorem]{Lemme}
\newtheorem{corollary}[theorem]{Corollaire}
\newtheorem{proposition}[theorem]{Proposition}
\newtheorem{example}[theorem]{Exemple}

\theoremstyle{remark}
\newtheorem{remark}{Remarque}

\begin{document}
	
	\hypersetup{pdfborder=0 0 0}
	\begin{titlepage}
		\newcommand{\HRule}{\rule{\linewidth}{0.5mm}}
		\begin{center}
			
			
			
			\vspace{1cm}
			
			\HRule \\ [0.8cm]
			{\Large {Rapport de stage :} }\\ [0.8cm]
			{\huge \bf Travail d'étude et de validation d'un code LES/DES et comparaison avec la théorie de la turbulence de von Kármán.} \\ [0.4cm]
			\HRule \\ [2cm]
			{\large \textbf{Julien TENAUD}} \\ [0.3cm]
			{ Étudiants en $2^{\text{ème}}$ année à \textsc{l'enseirb-matmeca}} \\ [0.2cm]
			{ Spécialité : Mathématiques appliquées et mécanique} \\ [1cm]
			{\small sous la direction de }\\ [0.6cm]
			{\large \textbf{Pr. F.Bertagnolio}} \\ [0.3cm]
			{ Senior scientist at \textsc{dtu wind and energy systems}} \\ [1cm]
			{\small avec pour tuteur universitaire }\\ [0.6cm]
			{\large \textbf{M. N.Barral}} \\ [0.3cm]
			{ Maître de conférence à l'\textsc{enseirb matmeca}}\\ [0.3cm]
			
			\vfill
			
			\begin{minipage}[c]{.4\textwidth}
				\begin{center}
					\includegraphics[width=1\textwidth]{logo_emkk.jpg}
				\end{center}
			\end{minipage}
			
			\vfill
			
			
			{Année 2023/2024}
			
			
		\end{center}
	\end{titlepage}
	
	\newpage
	
	\section*{Sommaire}
	
	\noindent\rule{\textwidth}{0.3pt}
	\vspace{0.5cm}
	
	\shadowbox{%
		\begin{minipage}{1\textwidth}
			\tableofcontents
		\end{minipage}
	}
	
\newpage
.
\newpage
	
	
	\section{Introduction}
	\subsection{Motivations et plan}
	
	Les éoliennes font maintenant partie de notre paysage. Elles sont de plus en présente sur terre (onshore) et maintenant en mer (offshore). En effet le besoin en énergie électrique est grandissant alors que nous somme en pleine crise écologique. Mais l'implentation et la recherche de performances des éoliennes n'est pas sans encombre. Exposition au vent, taille, raccordement au réseau... Les enjeux sont nombreux. Ce travaille s'incrit dans l'obectif de diminution du bruit généré par les pales d'éolienne. Le domaine d'étude et celui de l'aéro-acoustique. Dans de nombreux pays des normes gouvernementales sur le bruit émis [{\bf ref}] sont mises en place dans le bute de ne pas déranger les habitants aux alentours d'un parc éolien. Bien qu'aucune étude scientifique n'ai prouvé un quelconque effet nocif du bruit éolien sur la santé, de nombreuses plaintes ont été enregistrées de gens dérangés par ce bruit. Alors pour éviter que nos géants de métal ne devienne un ennemie publique, la réduction de leur effet sonore est nécessaire. \\
	
	Pour appréhender ce problème il faut d'abord comprendre comment les éoliennes génèrent du bruit. La grande majorité du son émis viens des pales de l'éolienne. En effet quand le flux d'air rencontre une pale il est perturbé et comme il y a interaction entre un fluide et une structure alors il y a apparition d'une couche limite. Dans notre cas cette dernière est turbulente. En effet le bout de pâle se déplace en moyenne à $80 m/s$ si on considère une éolienne classique. Dans l'étude de l'écoulement on à donc des Raynolds très élevés, de l'ordre de plusieurs milliers. La grande perturbation du flux d'air va engendrer la formation, tout autour de la pale, d'onde de pression qu'on appelle plus communément le bruit. Là où le son générer et le plus intense c'est au niveau du bord de fuite et plutôt en bout de pâle. Il y a un lien de causalité directe entre turbulence et son émis. En revanche il est considéré que le son émis n'a pas d'impacte sur la turbulence. C'est pourquoi la couche limite et le sillage turbulent des pâles sont étudiés.\\
	
	L'étude que j'ai mené s'incrit dans le travail de développement d'un code de calcul couplant l'aérodynamique et l'acoustique. Mon travail ne s'intéraissera qu'à la partie aérodynamique et plus précisément l'étude statistique d'une couche limite turbulente dans un canal parallélépipédique. Un code LES (Large Eddy Simulation) développé à \textit{DTU} permet un relevé de vitesse et de pression dans une simulation à la manière d'une expérience. À noté que celui-ci est très utile vû le coût d'une expérience en soufflerie. Afin d'étudier et de valider les résultats obtenu après calcul j'effecturai une analyse statistique notamment dans le domaine spectral de la turbulence et principalement des variables de vitesse. \\
	
	Plusieurs axes de recherche sont envisagés~: 
	\begin{itemize}
		\item Étudier l'impact du raffinement du maillage sur la turbulence et évaluer a priori l'impact sur des calculs liés à l'acoustique.
		\item Validation des hypothèses de modèles semi-empiriques incluant la reconstitution du tenseur spectral avec les données a partir d'approche RANS.
		\item Étudier l'impact de certaines hypothèse faites dans les modèles semi-empiriques : 
		\begin{itemize}
			\item turbulence isotrope dans la couche limite
			\item distribution de l'énergie cinétique turbulente sur les 3 composantes et son évolution à travers la couche limite
			\item validité de l'hypothèse de turbulence figée (fluctuations turbulents constantes au cours du temps)
			\item évaluer des loi plus générales que la théorie de la turbulence isotropique.
		\end{itemize}
	\item Si le temps le permet le projet pourrait déboucher sur des calculs couplant la théorie de la turbulence LES/DES avec des calculs (aéro-)acoustique (à bas Reynolds pour limiter les temps de calculs).
	\end{itemize}

	\subsection{Time line} 
	
	\begin{itemize}
		\item 18/06 : Théorie et pris en main des premiers calculs
		\item 6/07 : Préparation d'un worklow pour analyser les résultats (en python)
		\item 19/07 : Résultats et analyse (courbes)
		\item 30/07 : Rapport
	\end{itemize}

\section{Approche statistique et spectrale de la turbulence}

	Dans cette section nous introduiront mathématiquement certains outils d'étude statistique de la turbulence notamment dans le domaine spectrale . Nous nous pencherons sur les spectres en énergie et les fonctions de corrélation.
	
	\subsection{Fonction de corrélation et ses propriétés}
		
		Dans l'analyse d'un problème de turbulence on utilise la décomposition de Reynolds afin de décrire chaque grandeur avec un terme moyen et un terme de fluctuation, en prenant en exemple le champ de vitesse, 
		
		\begin{equation}
			U_i=<U_i> + u_i 	
		\end{equation}
	
		avec $<U_i>$ la vitesse moyenne (moyenne de Reynolds) et $u_i$ les fluctuations. \\
		Pour simplifier les équations nous utiliserons la notation d'Einstein de sommation implicite. \\
		La fonction de corrélation $R_{ij}$ se définie pour deux points $\textbf{x}$ et $\textbf{x'}$ ou deux temps $t$ et $t'$  tel que,
		
		\begin{equation}
		\begin{split}
			&R_{ij}(\textbf{x},\textbf{x'},t) = <u_i(\textbf{x},t), u_j(\textbf{x'},t)> \\
			&R_{ij}(\textbf{x},t,t') = <u_i(\textbf{x},t), u_j(\textbf{x},t')>
		\end{split}
		\end{equation}
	
		Dans le cas d'une turbulence homogène, c'est à dire qui conserve ses propriétés statistiques pour tout déplacement dans l'espace, alors $R_{ij}$ dépend uniquement de $\textbf{r}=\textbf{x}-\textbf{x'}$,
		
		\begin{equation}
			R_{ij}(\textbf{r},t) = <u_i(\textbf{x},t), u_j(\textbf{x'},t)>
		\end{equation}
	
		Nous pouvons définir la fonction d'autocorrelation,
		
		\begin{equation}
			R_{ii}(0,t) = <u_i(\textbf{x},t), u_i(\textbf{x},t)>
		\end{equation}
	
		\begin{proposition}[Indépendance statistique asymptotique]
			Quand $r\rightarrow\infty$, $R_{ij}(\textbf{r},t)\rightarrow0$. Plus les points considérés sont éloignés plus ils sont décorrélés.
		\end{proposition}
	
		\begin{proposition}
			Quand $t\rightarrow\infty$, $R_{ii}(0,t)\rightarrow0$. Un tourbillon se décorrèle de lui même au court du temps.
		\end{proposition}
	
		Définissons à présent une fonction qui nous intéressera tout au long de notre étude, le spectre qui est la transformée de Fourier tridimensionnelle de la fonction de corrélation,
		
		\begin{equation}
			\phi_{ij}(\textbf{k}, t) = \frac{1}{(2\pi)^3}\int R_{ij}(\textbf{r},t)e^{-ikr}d^3\textbf{r}
		\end{equation}
	 
	
		où $\textbf{k}$ est le vecteur d'onde. En prenant la transformée de Fourier inverse on peut obtenir :
		
		\begin{equation}
			R_{ij}(\textbf{r}, t) = \int \phi_{ij}(\textbf{k},t)e^{ikr}d^3\textbf{k}
			\label{spectra_space}
		\end{equation}
		
		Maintenant en imposant $i=j$ et $\textbf{r}=0$ dans (\refeq{spectra_space}) on obtient, 
		
		\begin{equation}
			<u_iu_i>=R_{ii}(0,t)=\int \phi_{ij}(\textbf{k},t)d^3\textbf{k}
			\label{verif}
		\end{equation}
	
		Alors il est possible de calculer le spectre en énergie nommé TKE par unité d'intensité de nombre d'onde (turbulent kinetic energy). 
		
		\begin{equation}
			E(\kappa)=2\pi \kappa^2\phi_{ii}~~[m^{3}.s^{-2}]
			\label{energie_spectra}
		\end{equation}
	
		où $\kappa = |\textbf{k}|$. On peut tracer le spectre d'énergie turbulente $log(E)=f(log(k))$. Ce spectre vérifie d'après (\refeq{energie_spectra}) et (\refeq{verif}),
		
		\begin{equation}
			<u_iu_i>=2\int_{0}^{\infty}E(\kappa,t)d\kappa
		\end{equation}
	
	\subsection{Spectre unidimensionnel et propriété de turbulence figée}
	
		On considère l'espace représenté par le repère cartésien $(O,x_1,x_2,x_3)$. L'écoulement turbulent étudié est homogène dans la direction \textbf{x} de l'espace. Alors on défini la corrélation spatial à deux points \\$R_{ij}(r_1, x_2,x_3,x_2',x_3',t) = <u_i(\textbf{x},t)u_j(\textbf{x'},t)>$ où $r_1=x_1-x_1'$. Le spectre spatial unidimensionnel est donc donné par,
		
		\begin{equation}
			\phi_{ij}^{[1]}(k_1,x_2,x_3,t)=\frac{1}{2\pi}\int_{-\infty}^{\infty}R_{ij}(r_1,x_2,x_3,x_2,x_3,t)e^{-ik_1r_1}dr_1
		\end{equation}
	
		où on a considéré $\textbf{x}$ et $\textbf{x'}$ sur une même ligne de direction $x_1$. Alors, si on considère $x_1'=x_1$ ($r_1=0$) et que $i=j$ on peut montrer avec une transformée de Fourier inverse que:
		
		\begin{equation}
			<u_iu_i>=2\int_{0}^{\infty}\phi_{ii}^{[1]}(k_1,x_2,x_3,t)dk_1
		\end{equation}
	
		À présent considérons la fonction de corrélation en temps $R_{ij}(\textbf{x},\tau)=<u_i(\textbf{x},t)u_i(\textbf{x},t')>$ au même point mais à deux temps différent. Il est alors possible d'obtenir la fonction spectral en fréquence :
		
		\begin{equation}
			\psi_{ij}(\omega,\textbf{x})=\frac{1}{2\pi}\int_{-\infty}^{\infty}R_{ij}(\textbf{x},\tau)e^{i\omega\tau}d\tau
		\end{equation}
	
		où $\omega$ représente les fréquences. On peut obtenir de même : 
		
		\begin{equation}
			<u_iu_i>=\int_{-\infty}^{\infty}\psi_{ij}(\omega,\textbf{x})d\omega
		\end{equation}
		
		
		Expérimentalement, les mesures de vitesse d'écoulement à plusieurs points au même moment ne sont pas facile à effectuer. On aimerai bien pouvoir relever les caractéristique de la turbulence en un point à plusieurs temps et en déduire des informations sur l'évolution des structures spatial. C'est pourquoi on fait l'hypothèse suivante:
		
		\begin{proposition}[Hypothèse de turbulence figée ou de Taylor]
			On considère que les structures spatiales d'intéret ne change pas de manière signifiante durant le temps de leur passage le long de la ligne de mesure. 
		\end{proposition}
	
		Ainsi, on suppose que l'échelle de temps liée à la dynamique des tourbillons est grande devant le temps de parcours du fluide dans la cavité de mesure. Dans le cas d'un écoulement de direction préférentiel $x_1$ cela peut s'exprimer tel que :
		
		\begin{equation}
			u_i=u_i(x_1-<U_i>t,x_2,x_3)
		\end{equation}
	
		Un des aspect de cette étude sera de vérifié si cette hypothèse est vérifiée dans le cas des calculs LES effectués. \\
		
		Dans les écoulements stationnaire où cette hypothèse est vérifié on peut alors montrer que :
		
		\begin{equation}
			\phi_{ij}^{[1]}(k_1,x_2,x_3)=|<U_1>|\psi_{ij}(<U_1>k_1,x_2,x_3)
		\end{equation}
	
		Cette expression relie le spectre en fréquence et le spectre unidimensionnel en nombre d'onde. Cela met en évidence la relation $\omega=<U_1>k_1$ entre la fréquence et le nombre d'onde unidimensionnel pour des composantes de Fourier convéctés à une vitesse moyenne $<U_1>$. 
		
		\subsection{Spectre bidimensionnel}
			
\section{Résultats théorique à partir du modèle de von Kármán de la turbulence}
		
	Dans cette section nous allons définir certaines des expressions des spectres en utilisant les résultat de von Kármán en 1948 \textbf{[\color{blue}ref von Karman 1948]} sans pour autant en dresser une liste exostive. Par la suite l'objectif sera de comparer le modèle de von Kármán théoriques avec les données issues des calculs LES à Reynolds variables. \\
	
	Le modèle utiliser par von Kármán vient de l'expression du spectre d'énergie (\refeq{energie_spectra}) sous l'hypothèse que la turbulence est isotropique \textbf{[\color{blue}ref Keith Wilson]},
	
	\begin{equation}
		E(\kappa) = \frac{4\Gamma(\nu + 5/2)}{3\sqrt{\pi}\Gamma(\nu)}\frac{\sigma^2\kappa^4l^5}{(1+\kappa^2l^2)^{\nu + 5/2}}
		\label{energie_spectra}
	\end{equation}

	où $\sigma^2$ est la variance, $\kappa$ est l'intensité du vecteur d'onde, $l$ "the length scale", $\Gamma()$ est la fonction gamma et $\nu$ est un paramètre qui définie la loi de puissance dans la zone inertiel ($\kappa l \gg 1$). Pour notre étude il sera fixé à $1/3$ afin de retrouver la loi de puissance de Kolmogorov de $-5/3$ \textbf{[\color{blue}ref Kolmogorov]}. \\

	De cette expression il est possible d'obtenir les spectres 'streamwise' avec une intégration par rapport à $k_2$ et $k_3$ des spectres d'énergie $E_{11}(\kappa)$, $E_{22}(\kappa)$ et $E_{33}(\kappa)$:
	
	\begin{equation}
		\phi_{11}^1(k_c,z) = \frac{36\Gamma(17/6)\sigma_1l}{55\sqrt{\pi}\Gamma(1/3)}\frac{1}{(1 + (k_cl)^2)^{5/6}}
	\end{equation}

	\begin{equation}
		\phi_{22}^1(k_c,z) = \frac{6\Gamma(17/6)\sigma_2l}{55\sqrt{\pi}\Gamma(1/3)}\frac{(3+8(k_cl)^2)}{(1 + (k_1l)^2)^{11/6}}
	\end{equation}

	\begin{equation}
		\phi_{33}^1(k_c,z) = \frac{6\Gamma(17/6)\sigma_3l}{55\sqrt{\pi}\Gamma(1/3)}\frac{(3+8(k_cl)^2)}{(1 + (k_1l)^2)^{11/6}}
	\end{equation}

	avec $\sigma_i = <u_iu_i>$, $l=C\frac{k_T^{3/2}}{\varepsilon}$, $k_c = \frac{\omega}{U_c}$.\\
	
	Ce sont ces trois spectres que nous chercherons à comparer avec ceux obtenu à partir des données issus des modèles LES dans la section $9$. Nous expliquerons donc dans la dites section comment nous obtenons les valeurs de $l$ et $k_c$.
	
	
		 
	
\section{Modèles WMLES et WRLES}

\section{Analyse simple des variables de vitesse turbulente}

\section{Validation de l'hypothèse de turbulence figée}

\section{Analyse spectral des variables turbulentes}

\section{Détermination du coefficient de décorrélation}

\section{Comparaison des spectres avec la théorie de von Kármán}



	
	
	
	
	
	
\end{document}